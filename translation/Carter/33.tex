Immediately prescribe some character and form of conduce to yourself, which you
may keep both alone and in company.

Be for  the most part  silent, or  speak merely what  is necessary, and  in few
words. We may, however, enter,  though sparingly, into discourse sometimes when
occasion calls for  it, but not on  any of the common  subjects, of gladiators,
or  horse  races, or  athletic  champions,  or  feasts,  the vulgar  topics  of
conversation; but principally not of men, so  as either to blame, or praise, or
make comparisons.  If you are able,  then, by your own  conversation bring over
that of your company  to proper subjects; but, if you happen  to be taken among
strangers, be silent.

Don't allow your laughter be much, nor on many occasions, nor profuse.

Avoid swearing, if possible, altogether; if not, as far as you are able.

Avoid public and  vulgar entertainments; but, if ever an  occasion calls you to
them, keep  your attention  upon the  stretch, that  you may  not imperceptibly
slide into  vulgar manners. For be  assured that if  a person be ever  so sound
himself, yet, if his  companion be infected, he who converses  with him will be
infected likewise.

Provide things relating to  the body no further than mere  use; as meat, drink,
clothing, house, family. But strike off  and reject everything relating to show
and delicacy.

As far as possible, before marriage, keep yourself pure from familiarities with
women, and,  if you indulge  them, let it be  lawfully. But don't  therefore be
troublesome  and  full of  reproofs  to  those  who  use these  liberties,  nor
frequently boast that you yourself don't.

If anyone tells  you that such a  person speaks ill of you,  don't make excuses
about what is said of you, but answer: ``He does not know my other faults, else
he would not have mentioned only these.''

It is not necessary  for you to appear often at public  spectacles; but if ever
there is a  proper occasion for you  to be there, don't  appear more solicitous
for anyone than for yourself; that is, wish things to be only just as they are,
and him only  to conquer who is the  conqueror, for thus you will  meet with no
hindrance.  But abstain  entirely from  declamations and  derision and  violent
emotions. And  when you  come away, don't  discourse a great  deal on  what has
passed, and what does not contribute to your own amendment. For it would appear
by such discourse that you were immoderately struck with the show.

Go not [of your own accord] to  the rehearsals of any (authors), nor appear [at
them] readily. But, if you do appear,  keep your gravity and sedateness, and at
the same time avoid being morose.

When  you are  going to  confer with  anyone, and  particularly of  those in  a
superior station,  represent to yourself how  Socrates or Zeno would  behave in
such a case, and you will not be at a loss to make a proper use of whatever may
occur.

When you are  going to any of  the people in power, represent  to yourself that
you will not  find him at home; that  you will not be admitted;  that the doors
will not  be opened to you;  that he will take  no notice of you.  If, with all
this, it is  your duty to go,  bear what happens, and never  say [to yourself],
``It was  not worth  so much.'' For  this is  vulgar, and like  a man  dazed by
external things.

In parties of conversation, avoid a  frequent and excessive mention of your own
actions and  dangers. For, however agreeable  it may be to  yourself to mention
the risks  you have run,  it is  not equally agreeable  to others to  hear your
adventures. Avoid,  likewise, an  endeavor to  excite laughter.  For this  is a
slippery point, which  may throw you into vulgar manners,  and, besides, may be
apt to  lessen you in the  esteem of your acquaintance.  Approaches to indecent
discourse are  likewise dangerous. Whenever,  therefore, anything of  this sort
happens, if there  be a proper opportunity, rebuke him  who makes advances that
way; or, at least, by silence and blushing and a forbidding look, show yourself
to be displeased by such talk.
