THE ENCHEIRIDION OF EPICTETUS.1

by T.W. Rolleston 1881, London: Kegan Paul, Trench & Co.

*Note: The chapter numbering has been changed to using all numeric references, Rolleston used both numeric for chapters and Greek letters for sections. Some Britsh spellings have been changed to American. There may still be errors in this work. Send an email to louis<at>letsreadgreek<dot>com and let me know so I can fix it. This work is freely available to all to use; no attribution is necessary.

 

 1In the following translation I have in the main followed Schweighauser’s text, ed. 1798; but I have sometimes adopted another fairly supported reading which seemed to me to fall in more with the general line of Epictetus’ thought. I cannot confidently suppose that there are no blunders in it, but I hope and think that there are few seriously misleading ones.

  1.1   OF things that exist, some depend upon ourselves,2 others do not depend upon ourselves. Of things that depend upon ourselves are our opinions and impulses, desires and aversions, and, briefly, all that is of our own doing. Of things that do not depend upon ourselves are the body, possessions, reputation, civil authority, and, briefly, all that is not of our own doing.

  2ἐφ' ἡμῖν ἐστίν. The full sense which Epictetus puts into these words cannot be briefly rendered in English.

  1.2    And the things that depend upon ourselves are in their nature free, not liable to hindrance or embarrassment, while the things that do not depend upon ourselves are strengthless, servile, subject, alien.

  1.3   Remember then, that if you take things which are by nature subject, to be free; and things alien to be your proper interest, you will be embarrassed, you will bewail yourself, you will be troubled, you will blame Gods and men. But if you consider that only to be yours which truly is so, and the alien as what it is, alien, then none shall ever compel you, none shall hinder you, you will blame no one, you will accuse no one, you will not do the least thing reluctantly, none shall harm you, you will have no foe, for you will suffer no injury.

  1.4    Aiming then at things so high, remember that it is no moderate passion wherewith you must attempt them, but you must utterly renounce some things, and put some, for the present, aside, For if (let us say) you make this also your object, to attain a position of authority and wealth, then you not only run the risk (through aiming at the first things too) of failing to gain these ends, but will most assuredly miss those others through which alone freedom and happiness are born,

  1.5   Straightway, then, practice saying to every harsh-seeming phantasm, You are a Phantasm, and not by any means the thing you appear to be. Then realize it, and test it according to the crIterions you possess; but especially by this supreme criterion, whether it concerns anything that depends upon ourselves, or something that does not depend upon ourselves, And if the latter, then be the thought instantly at hand, It is nothing to me,

  2.1    REMEMBER that desire announces the aim of attaining the thing desired, and aversion that of not falling into the thing shunned; and that to miss what you desire is unfortunate, but it is misfortune to fall into what you shun, But you can never fall into anything that you shun, if you will shun only things contrary to Nature which lie within your power: but if you shun disease, or death, or poverty, you will have misfortune, Withdraw then your aversion from those things that do not depend upon ourselves, and place it upon those things contrary to Nature which do depend upon ourselves,

  2.2     And let desire, for the present, be utterly effaced; for if you are desiring something of the kind that does not depend upon ourselves, it must needs be that you miscarry,and of the things that do depend upon ourselves, of such as you may fairly desire, none are yet open to you. Use therefore only [tentative] advances and withdrawals, and that but lightly, and with exception, and indifferently.

  3    IN the case of everything that allures the mind, or offers an advantage, or is beloved by you, remember, from the least thing upward, to think of it in its true nature. For instance, if you like an earthen jar, think I like an earthen jar, for so your mind will not be confounded if it should break.  And if you love your child or your wife, think I love a mortal and so you will not be confounded when they die.

  4    WHEN you are about to take in hand some action, bethink yourself what it really. Is that you are about to do If you propose to go to the bath, represent to yourself all the things that take place at the bathing establishment, the squirting of water, the beating, the bad language, the theft. And after this fashion you will take the matter more safely in hand if you say I intend simply to bathe, and to maintain my purpose according to Nature. And similarly with every action. For thus if anything’ should occur to cross you in your bathing, you will instantly think I did not only intend to bathe, but also that my purpose should be maintained according to Nature. But it will not be so maintained lf I let myself be vexed at what occurs.

  5    IT is not things in themselves, but the opinions held about them which trouble and confuse our minds. Thus, Death is not really terrible — if it were so it would have appeared so to Socrates— but the opinion about Death, that it is terrible, that it is wherein the terror lies. Wherefore when we are hindered, or confounded, or grieved, let us never cast the blame upon others, but upon our selves; that is, on our opinions of things. A man untaught in philosophy1 accuses others on the score of his misfortunes; he who has begun to be taught accuses himself; he who is fully taught, neither others nor himself.

  1 ἀπαίδευτος, a word including the ideas both of teaching and training.

  6    BE not puffed-up on account of any that is not of yourself.  If your horse were proud and should say, I am beautiful, that would be tolerable.  But when you are proud, and say, I have a beautiful horse, how that it is an excellence in your horse that you are proud of.  What then is really your own?  This, to make use of the phantasms. So that when you deal according to Nature in your use of the phantasms,  then you may pride yourself, for then you will be priding yourself on an excellence which is really your own.

  7    EVEN as in a sea-voyage, when the ship is brought to anchor, and you go out to fetch in water, you make a bye-work of gathering a few roots and shells upon the way, but have need ever to keep your mind fixed upon the ship, and constantly to look round lest at any time the master of the ship should call, and must, if he call, cast away all those things, lest you be treated like the sheep that are bound and thrown into the hold: So it is with human life also, and nothing hinders the comparison if there be given wife and children instead of shells and roots.1 And if the master call, run to the ship, forsaking all those things and looking not behind.  And if you be in old age, go not far from the ship at any time, lest the master should call, and you should not be ready.

  1 Or ‘if there be given wife and children,  &c..,, nothing hinders [you from taking them].’

   8    DO not seek to have all things happen as you would choose them, but rather choose them to happen as they do; and so shall the current of your life flow free.

   9    DISEASE is a hindrance of the body, not of the will, unless the will itself consent. Lameness is a hindrance of the leg, not of the will. And this you may say upon every occasion, for nothing can happen to you but you will find it a hindrance, not of yourself, but of some other thing.

   10   REMEMBER, at anything that shall befall you, to turn to yourself, and seek what faculty you possess for making use of it.  If you see a beautiful person, you will find a faculty for that, namely, Self- mastery. If toil is laid upon you, you will find the faculty of Perseverance If you are reviled, you will find Patience, And making this your wont, you will not be rapt-away by the phantasms.

   11   NEVER in any case say I have lost such a thing, but I have returned it. Is your child dead? it is a return. Is your wife dead? it is a return.  Are you deprived of your estate? is not this also a return? But he who deprives me of it is wicked! But what is that to you, through whom the giver demands his own! As long therefore as he grants it to you, steward it like another’s property, as travelers use an inn.1

   1 ‘Steward it’ (ἐπιμελοῦ) expresses one characteristic of Epictetus’ view of life, ‘as travelers use an inn,’ another; no real comparison is intended.

   12.1   IF you wish to advance in philosophy you must dismiss such considerations as If I neglect my affairs, I shall not have wherewithal to support life. If I do not correct my servant, he will be good-for-nothing, For it is better to die of hunger,having lived without grief and fear, than to live with a troubled spirit amid abundance. And it is better to have a bad servant, than an afflicted mind.

   12.2   Make a beginning then with small matters. Is a little of your oil spilt? or a little wine stolen? Then say to yourself For so much, tranquility of mind is bought, this is the price of peace.  For nothing can be gained without paying for it, And when you call your servant, reflect that he may not hear you, or that hearing, he may not do your bidding. For him indeed that is not well, but for you it is altogether well that be have not the power to disturb your mind.

   13   IF you wish to advance, you must be content to let people think you senseless and distraught as regards external things.  Wish not ever to seem wise,  and if ever you shall find yourself account to be somebody, then mistrust yourself. For know that it is not an easy matter to make a choice that shall agree both with external things and with Nature, but it must needs be that he who is careful of the one shall neglect the other.

   14.1   You are quite astray if you desire your wife and children and friends to live for ever, for then you are desiring things that are not of yourself to be of yourself, and to control that which is not in your power.  So also if you desire that your servant never should display any shortcomings, you are a fool;  for that is as much as to desire that imperfection should not be imperfection, but something else. But If you wish never to fall short of your desires, this indeed is possible to you; this therefore practice, namely, the practicable.

   14.2    You own a master when another has power over the things that are pleasing or displeasing to you, to give them or to take them away. Whosoever then would be free, let him neither desire nor shun any of the things that depend not upon himself; otherwise he must needs be enslaved.

   15   BEAR in mind that you should conduct yourself in life as at a feast. Is some dish brought to you? Then put forth your hand and help yourself in seemly fashion. Does it pass you by? Then do not hold it back. Has it not yet come to you? Then do not stretch out for it at a distance, but wait till it is at your hand. And thus doing with regard to children, and wife, and authority, and wealth, you will be a worthy guest at the table of the Gods, And if you even pass over things that are offered to you, and refuse to take of them, then you will not only share the banquet of the Gods, but also their dominion. For so doing, Diogenes and Heracleitus and such as they were rightly divine, as they were said to be.

   16   WHEN you shall see one lamenting in grief because his son is gone abroad, or because he has lost his wealth, see to it that you be not rapt-away by the phantasm, to think that he has suffered a real misfortune in external matters. But be the thought at hand, It is not the fact itself which afflicts this man—since there are others whom it afflicts not—but the opinion he has conceived about it. And do not hesitate as far as words go, to give him your sympathy, and even, if so it be, to lament with him. But take heed that your lamenting be not from within.

   17   REMEMBER that you are an actor in a drama, of such a part as it may please the master to assign you, for a long time or for a little as he may choose.  And if he will you to take the part of a poor man, or  a cripple, or a ruler, or a private citizen, then may you act that part with grace! For to act well the part that is allotted to us, that indeed is ours to do, but to choose it is another’s,

   18   IF a raven croaks you a bad omen, be not rapt away by the phantasm, but straightway make a distinction with regard to yourself1 and say, This bodes something perhaps to this poor body or this little property of mine, or to my reputation, or my wife or children, but to myself, nothing.  For to me, if I will to have it so, all omens are fortunate.  And whatever of these things may come to pass, it lies with me to reap some benefit from it.

   1διαίρει παρὰ σεαυτῷ —distinguish what concerns your real self from what does not; or perhaps it merely means  ‘draw a distinction inwardly.’

   19.1   YOU may be always victorious if you will never enter into any contest where the issue does not  wholly depend upon yourself.

   19.2   When you see a man honored above others, or mighty in power, or otherwise in high repute, look to it that you esteem him not blessed, being  rapt-away by the phantasm.  For if Good, in its essence, be in those things which depend upon ourselves, then there is no place for jealousy or envy, and you yourself will not wish to be a general, or a prince, or a consul—but to be free.  And to this there is but one way — disdain of the things that not up ourselves.

   20   REMEMBER that it is not he who strikes you or he who reviles you who does you an injury, but the opinion you have about the things, that they are injurious.  Therefore when someone shall provoke you to wrath, know that you are provoked by your own conception. Strive then at the outset not to be rapt-away by the phantasm; for if you shall once succeed in gaining time and delay you will more easily master yourself.

   21   DEATH and exile and all things that appear terrible, let these be every day before your eyes.  But Death most of all, for so you will never feel any condition to be wretched, nor think any very greatly to be desired.

   22   IF you set your heart upon philosophy, you must straightway prepare yourself to be laughed at and mocked by many who will say Behold a philosopher  arisen among us! or How came you by that brow of scorn? But do you cherish no scorn, but hold to those things which seem to you the best, as one set by God in that place. Remember too, that if you abide in those ways, those who first mocked you, the same shall afterwards reverence you; but if you yield to them, you will be laughed at twice as much as before.

   23   IF ever it shall happen to you to be turned to things of the outside world in the desire to make yourself acceptable to someone, know that you have lost your position.1  Let it be enough for you in all things to be a philosopher. But if you desire also to seem one, then appear so to yourself, and it will suffice.2

   1 ἴσθι ὅτι ἀπώλεσας τὴν ἔνστασιν.

   2According to another reading, ‘and you will be able to do this,’

   24.1   NEVER afflict yourself with such reflections as I shall live without honor and never be anybody  anywhere. For if to live without hon our be really a misfortune, know that it is not possible for you to fall into misfortune, any more than into vice, through anything that another can do. Is it then of your own doing that you are appointed to the magistracy or invited to feasts? By no means, How then is this to be without hon our? and how do you say that you shall never be anybody, whose part it is to be somebody in those things only which depend upon yourself, and in which it is in your power to be of the highest worth?

   24.2    But then I can be of no service to my friends. How say you? of no service? Certainly they shall not have money from you, nor will you be able to get them made Roman citizens. But who told you that these things were of those that depend upon ourselves, and were not alien to us? And who is able to give that which be himself has not?

   24.3   Acquire then, they say, in order that we may possess, Well, if I am able to acquire, without a loss of modesty, or faith, or high-mindedness, show me the way, and I will do it. But if you require me to sacrifice that which I have, which is really good, in order that you may compass what is no good at all, behold how unjust and inconsiderate you are! And is it not better than money, to have faith and modesty in a friend? Then rather help me on my way, and take part with me, than require me to do anything which would compel me to renounce these things.

   24.4   But, you say, I shall not be bearing my part in the service of my country!  Again, what do you take service to consist in? Your country will not be enriched through you with porticos and public baths. And what then? The smith does not supply her with shoes, nor the shoemaker with weapons, but it is enough that every man fulfil his own work, And if you have made one faithful and modest citizen for her, are you then of no service? Wherefore neither shall you be useless to your country.

   24.5   What place then, you say, shall I hold in the State? Whatever place you are able to hold, guarding still your modesty and faith. But if you cast away these things in order to be of service to the State, of what service do you think you will be to her then, when you are perfected in the contrary qualities?

   25.1   Is  some one preferred before you at a feast or in salutation, or in being invited to give counsel? Then if these things be good, you should rejoice that he has gained them;  but if evil, why grieve that you have not? But remember  that if you do not do as other men, in order to gain the things that depend not upon ourselves, neither shall you be rewarded as they.

   25.2   For how is it possible for you to have an equal share with him who hangs about other men’s doors, and attends upon them, and flatters them, when you yourself will do none of these things? You are unjust then, and insatiable, if you wish to gain the things that depend not upon ourselves, for nothing, and without paying the price for which they are sold.

   25.3   But how much is your head of lettuce sold for? A penny perchance. Go to, then:  if one will lay out a penny he may have a head of lettuce; but you who do not choose to lay out your penny shall not have your lettuce. But you must not suppose that you will be therefore worse off than he. For he has the lettuce, but you the penny which you did not choose to part with.

   25.4    And in this matter also the same principle holds good. You are not invited to somebody’s banquet? That is because you did not give the entertainer the price that banquets are sold for — and they are sold for flattery, they are sold for attendance.  Pay then the price if you think you will profit by the exchange. But if you are determined not to lay out these things, and at the same time to gain the others—surely you are a greedy man, and an infatuated.

   25.5   Shall you have nothing then instead of the banquet which you give up? Yea verily, you shall have this — not to have praised one whom you did not care to praise, nor to have endured the insolence of a rich man’s doorkeepers.

   26   THE will of Nature is to be learned from matters in which we ourselves are not concerned.1  For instance, when a boy breaks a cup, if it be another man’s, you are ready enough to say, Accidents will happen. Know then, that  when your own is broken it behoves you to be as though it were another man’s. And apply this method even to greater things. When a neighbor’s wife or child dies, who is there that will not say, It is the lot of humanity. But when your own wIfe or child is dead, then it is straightway, Alas! wretched that I am! But you ought to remember how you felt when you heard of another in the same plight.

   1I think this is the true reading, although the other — ‘from matters in which we do not differ from each other ‘—has more external evidence to support it.

   27   AS a mark is not set up to be missed, even so it is not possible that there should be any soul of evil2 in the world.

   2 κακοῦ φύσις

   28   IF anyone were to expose your body in public, that every passerby might do as he liked with it, you would be indignant. Is it nothing to be ashamed of then that you should set your mind at the mercy of all the world, to be troubled and disturbed whenever anyone should happen to revile you?

   29.1   0F every work you take in hand to do, mark well the conditions and the consequences, and so enter upon it. For if you do not this, you will at first set out eagerly, not regarding what is to follow, but in the end thereof, if any difficulties have arisen, you will leave it off with shame.

   29.2   So you wish to conquer in the Olympic games, my friend? And I too, by the Gods, and a fine thing it would be! But first mark the conditions and the consequences, and then set to work. You will have to put yourself under discipline; to eat by rule; to avoid cakes and sweetmeats; to take exercise at the appointed hour whether you like it or no, in cold and heat; to abstain from cold drinks and from wine at your will; in a word, to give yourself over to the trainer as to a physician. Then in the conflict itself you are likely enough to dislocate your wrist or twist your ankle, to swallow a great deal of dust, to be severely thrashed, and, after all these things, to be defeated.

   29.3   If, having considered these circumstances, you are still in the mind to enter for the Olympic prize, then do so. But without consideration, you will turn from one thing to another like a child, who now plays the wrestler, now the flute player, now the gladiator, then sounds the bugle call, or declaims like an actor; and so you too will be first an athlete, then a gladiator, then an orator, then a philosopher, and nothing with your whole soul; but as an ape you will mimic every sight you see, and flatter yourself with one thing after another. For you approached nothing with consideration, nor with systematic diligence, but lightly, and with but a cold desire.

   29.4   And thus some men, after having seen a philosopher and heard discourse like that of Euphrates (yet who indeed can say that any discourse is like his?) desire that they also may become philosophers.

   29.5   But O man —  first consider what it really is that you are desiring to do, and then inquire of your own nature, whether you have power to support the undertaking. Do you desire to become a pentathlos1 or a wrestler? Then scan your arms and your thighs and try the strength of your loins. For nature endows different men with different capacities.

   1Pentathios, a champion in five athletic exercises, viz. running, leaping, throwing the quoit, throwing the javelin, wrestling.

   29.6    And do you think that you can be a sage and at the same time continue to eat and drink and indulge your desires and be fastidious, just as before? Nay verily, for you must watch and labour, and withdraw yourself from your household and be despised by any serving-boy and be ridiculed by your neighbors, and you must take an inferior position in all things, in reputation, in authority, in courts of justice, in dealings of every kind.

   29.7   Consider these things; whether you are willing at such a price to gain serenity, freedom, and immunity from vexation. And if not, renounce that aim at once, and do not like a child at play be now for a little a philosopher, then a tax-gatherer, then a public speaker, then a procurator of the Empire.  For these things do not agree among themselves, and, good or bad, it behoves you to be one man. You must either cultivate external things or your own essential part, you must show your skill in the management of either your outward or your inward life — in short, you must take up the position either of a sensualist1 or of a sage.

   1Whever Epictetus uses the word ἰδιώτης, I ha rendered it by Sensualist, ‘ἰδιώτης originally meant simply a private citizen, one who took no part in the government of the State, Epictetus uses it somewhat in the sense of ‘laymen,’ one who is of the world, not of the philosophic guild; the distinction being that the ἰδιώτης  looks for happiness, not in the things of the soul, but in relations with the external world established by means of the senses. On this distinction I venture to found my translation of the word.

   30   0UR duties are universally fixed by the relations in which we find ourselves.  Is such a man your father? Then your duty is to bear dictation from him, to take care of him, to yield to him in all things, to be patient with him when be chides you, when he beats you. But if he be a bad father?  Is there then a natural law that you should be related by kin to a good father? No, but simply to a father.  And does your brother do you wrong? Then guard your own proper position towards him, and do not consider what he is doing, but rather what you may do in order that your will may be in accordance with Nature. For no other can ever harm you if you do not choose it yourself, but you are harmed then, when you imagine yourself to be so. Thus then you may discover your duties by considering those of a neighbor, of a citizen, of a general, —if you will accustom yourself to observe the relationships.

   31.1   IF religion towards the Gods, know that the principal element is to have right opinions about them, as existing, and as managing the Whole in fair order and justice; and then to set yourself to obey them, and to yield to them in all that happens and to follow it willingly, as accomplished under the highest counsels. For thus you will never at any time blame the Gods, nor accuse them of being neglectful.

   31.2   But this can be done in no other way than by placing good and evil in the things that depend upon ourselves, and withdrawing them from those that do not; for if you take any of the latter things for good or evil, then when you fail to obtain what you desire, and fall into what you would avoid, it Is inevitable that you must blame and hate those who caused you to do so.

   31.3   For it is the inherent nature of every creature to fly from and shun all things that appear harmful, and the causes of the same, but to follow after and admire things, and the causes of things, which appear beneficial. It is impossible then that one who thinks himself harmed should take delight in that which seems to harm him, just as it is impossible that he should take delight in the very injury itself.

   31.4    And thus it is that a father is reviled by his son when he will not give him a share of the things that are thought to be good. And it was this that set Polyneices and Eteocles at war with each other, the opinion, namely, that royalty was a good. And through this the Gods are blamed by the husband man and the sailor, by the merchant and by those who lose their wives or children.  For when advantage is, there also is religion. So that he who is careful to desire and to dislike as he ought is careful at the same time of religion.1

   1εὐσέβεια is the Greek word which I have translated by ‘religion’ in this chapter. It is derived from σέβομαι, which means to have a pious regard, reverent respect or awe, towards some person or thing, God or man, or ancient custom, &c. Epictetus seems to think that a man may have a good religion or a bad one. If he cares only for material pleasures, he will have respect, enthusiasm, ‘religion,’ only for the sources of such pleasures. But if he desires what is truly good, he will necessarily have these feelings towards the Gods, i.e. the powers that make for goodness. Hence by rightly regulating his desires and aversions he ensures that his religion shall be rightly directed. Compare the Discourses of Epictetus, ii. 23 (Mr. Long’s translation),  ‘For universally, be not deceived, every animal is attracted to nothing so much as its own interest. . . . this is father, and brother, and kinsman, and country, and God;’ and the Biblical expressions ‘whose God is their belly,’‘worshippers of tables.’ In Disc. i. 27, Epictetus says, ‘Whenever religion and profit do not coincide, religion is doomed.’ This looks like an acquiescence in that ethical theory which we mostly think of in connection with his antagonists, the Epicureans, that Happiness is the raison d’etre of morality.

   31.5   But it is right too that every man should pour libations, and offer sacrifices, and offer firstfruits according to the customs of his fathers, purely, and not supinely, nor negligently, nor indeed scantily, yet not beyond his means.

   32.1   THEN you go to inquire of an oracle, remember that though you do not know beforehand what the event will be (for this very thing is what you have come to learn from the seer), yet of what nature it must be (if you are a philosopher), you knew already when you came.  For if it be of those things which do not depend upon ourselves, it follows of necessity that it can be neither good nor evil.

   32.2    When you go then to the seer, bring with you neither desire nor aversion, and approach him, not with trembling, but with the full assurance that all events are indifferent, and nothing to you, and that whatever may befall you it will be your part to use it nobly; and this no one can prevent.  Go then with a good courage to the Gods as to counselors, and, for the rest, when something has been counseled to you, remember who they are whom you have chosen for counselors, and whom you will be slighting if you are not obedient.

   32.3   Therefore, as Socrates insisted, you should consult the oracle in those cases only where your judgment depends entirely upon the event, and where you have no resources, either from reason or any other method, for knowing beforehand what is independently certain in the case. Thus, when it may behove you to share some danger with your friend or your country, do not inquire whether you may [safely] do so. For if the seer should announce to you that the sacrifices are inauspicious, that clearly signifies death, or the loss of some limb, or banishment; yet Reason convinces that even with these things you should stand by your friend and share your country’s danger. Mark therefore that greater seer, the Pythian, who cast out of the temple one who, when his friend was being murdered, did not help him.

   33.1   0RDAIN for yourself forthwith a certain principle and outline of conduct which you may observe both when you are alone and among men.

   33.2   And for the most part keep silence, or speak only what is necessary, and in few words.1 But when occasion shall require us to speak, then we shall speak, but sparingly, and not about any subject at haphazard, nor about gladiators, nor horse races, nor athletes, nor about things to eat or drink, which one hears talked about everywhere, but especially not about men, as blaming, or praising, or comparing them.

   1Nature has given men one tongue and two ears, in token that we should listen twice as much as we speak.’ (Fragment of Epictetus.)

   33.3   If then you are able, let your discourse draw that of the company towards what is fitting; but if you find yourself apart among strangers, keep silence.

   33.4   Do not laugh much, nor at many things, nor unrestrainedly.

   33.5   Refuse altogether to take an oath, if possible; if not, then as much as circumstances permit.

   33.6   Avoid banquets given by strangers and by the sensual1 But if you ever have occasion to go to them, then keep your attention on the stretch that you do not fall into sensuality. For know that if your companion be corrupted, you, who have conversation with him, must needs become corrupted also, even though yourself should chance to be pure.

   1Ἑστιάσεις τὰς ἔξω καὶ ιδιωτικὰς διακρούου see note on ἰδιώτης  Ench. 29.3 (p. 23)

   33.7   In things that concern the body, such as food, drink, clothing, habitation, servants, you must only accept what is absolutely needful. But all that makes for show or luxury, you must utterly proscribe.

   33.8    Concerning sexual pleasures, it is right to be pure before marriage, as much as in you lies, But if you do indulge in them, let it be according to what is lawful,2 But do not in any case make yourself disagreeable to those who use such pleasures, nor be fond of reproving them, nor of putting yourself forward as not using them.

   2 Ὡς νόμιμον.  Schweighauser's Latin version (with the reading ὧν νόμιμον) gives 'ea (venere) utere quai nihil flagitiosi habeat.' The whole sentence is 'Απτομένῳ δὲ, ὡς (ὧν) νόμιμον μεταληπτέον.

   33.9   If you shall be informed that some one has been speaking ill of you, do not defend yourself against his accusations, but answer, He little knew what other vices there are in me, or he would have said more than that.

   33.10   You need not go often to the arena; if how ever occasion should take you there, do not appear interested on any man’s side except your own; that is to say, desire that that only may happen which does happen, and that the conqueror may be be who wins; for so shall you not be embarrassed. But shouting, and laughter at this or that, and gesticulation, all this you must utterly abstain from.  And after you have gone away, do not talk much over what has passed, so far as it does not tend towards your own improvement. For from that it would appear that you bad been impressed with the spectacle.1

   1 ὅτι ἐθαύμασας τὴν θέαν.

   33.11   Do not attend everybody’s recitations1 nor be easily induced to go to them. But if you do go, preserve (yet without making yourself offensive) your gravity and tranquility.

   1Such recitations were common at Rome, when authors read their works and invited persons to attend (Long). Perhaps Epictetus disliked the flattery and self laudation which these occasions would give rise to.  

   33.12   When you are about to meet any person, especially if he be one of those considered to be high in rank, put before yourself what Socrates or Zeno would have done in such a case. And then you will not fail to deal fittingly with the occasion.

   33.13   When you are going to see one of those who are great in power, imagine that you will not find him at home, that you will be shut out, that the doors will be banged in your face, that he will take no notice of you. And if in spite of these things it be right for you to go, then go, and bear whatever may happen, and never say to yourself I did not deserve such treatment.2  For that is sensual, and shows that you are subject to vexation from external things.

   2Schweighauser gives ‘Non erat tanto’—’ it was not worth it.’ I have followed Politian’s version, Talia non merebar,’ The Greek is οὐκ ἦν τοσούτου.

   33.14   In company, be it far from you to bring your own doings and dangers constantly and disproportionately into notice. For though it is pleasant for you to tell of your own dangers, yet your adventures are not equally pleasant for other persons to hear.

   33.15   Be it far from you to move laughter. For that habit easily slips into sensuality; and it is always enough to lower your neighbor's respect for you.

   33.16   And it is dangerous to approach to vicious conversation. Therefore when anything of the kind may arise, rebuke him who approaches thereto, if you can do so opportunely. But if not, show at least by your silence, and blushing, and serious looks that his words are disagreeable to you.

   34   WHEN you have conceived the phantasm of some pleasure, guard yourself that you be not rapt-away by it, but delay with yourself a little and let the thIng await you for a while. And then bethink yourself of the two periods of time, the one in which you will enjoy the pleasure, the other, in which, after having enjoyed it, you will repent of it, and reproach yourself; and set over against this how you will rejoice and commend yourself if you have abstained. But if it shall seem fitting to you to do the thing, beware lest you have been conquered by the flattery and the sweetness and the allurement of it. But set on the other side how much better would be the consciousness of having won that victory.

   35   WHEN you are doing something which you have clearly recognized as being right to do, do not seek to avoid observation in the act, even though you should know that the multitude will form a wrong opinion about it, For if you are acting wrongly, then you should have avoided the action altogether. But if rightly, why fear those who wrongly rebuke you?

   36   AS the sayings It is day, It is night1, are perfectly justifiable if viewed disjunctively,2* but unjustifiable if viewed together, even so, at a feast, to pick out the largest portion for oneself may be justifiable if the act is viewed merely as it concerns the body, but is unjustifiable if viewed as it concerns the preservation of the proper community of the feast. Therefore when you are eating with another person, remember not merely to look at the value for the body of the things that are set before you, but to preserve also the reverence due to the giver of the feast.3*

   1Schweighauser says pathetically of this chapter: ‘In nullo Enchiridii capita tanta, quanta in hoc, librorum est discrepantia: in ea vero parte in qua mihi maxima inesse difficultas visa erat, in ipsa extremitate, miro modo iidem libri consentiunt.’ The text (according to Schweighauser’s reading) is as follows: Ὡς τὸ Ἡμέρα ἰστὶ, καὶ Νύξ ἐστι, πρὸς μὲν τὸ διεζευγμένον μεγάλην ἔχει ἀξίαν, πρὸς δὲ τὸ συμπεπλεγμένον ἀπαξίαν, οὕτω καὶ τὸ τὴν μείζω μερίδα ἐκλέξασθαι, πρὸς μὲν τὸ σῶμα ἐξέτω ἀξίαν, πρὸς δὲ τὸ, τὸ κοινωνικὸν ἐν ̔στιάσει, οἷον δεῖ, φυλάξαι, ἀπααξίαν ἔχει. Ὅταν οὖν συεσθίῃς ἑτέρῳ, μέμνησο, μὴ μόνον τὴν πρὸς τὸν ἑστιάτορα αἰδῶ φυλάξαι.

    The last two words are Schweighauser’s emendation for the perplexing reading οἷαν δεῖ φυλαχθῆναι, in which (with slight variations such as οἷον δ. φ., οἷαν σε δ. φ.) all the MSS. and old editions consent. One MS., strangely enough, has αἰδοῖ for οἷον δεῖ in the middle of the chapter.  I have given the plain sense of the chapter as well as I could, but it has many difficulties.

    2*(note 1, p. 37) ‘If viewed disjunctively,’ that is, if you say ‘It is day, or it is night,’

    3*(note 2, p. 37)Epictetus alludes to the feast of life, where ‘the Gods’ are the hosts. See ch, xv.

    37   IF you have assumed a part beyond your power to play, then you have not only come to shame in that, but have missed one which you could have thoroughly fulfilled.

    38   IN going about, as you are careful not to step upon a nail or twist your foot, even so be careful that you do no injury to your own essential part. And if we observe this we shall the more safely undertake whatever we have to do.

    39   THE measure of gain for every man is the body, as the foot is of the shoe. If you take this as your standpoint, you will preserve the measure. But if you go beyond it, you must thenceforward of necessity be borne, as it were, down a steep for the rest of the way. And so it is with the shoe; if you go beyond [what is proper for] the foot, you will have your shoe first gilded, then dyed purple, then embroidered. For there is no end to that which has once transgressed its measure.

    40   FROM the age of fourteen years upwards women are accustomed to the flattery of men (κυρίαι καλοῦνται).  Seeing then that there is nothing else open to them but only to serve the pleasure of men, they begin to beautify themselves and to place all their hopes in this, It is right then to study that they may perceive themselves to be valued for nothing else than modesty and decorum.

    41   IT is a sign of a dull nature1  to occupy oneself deeply in matters that concern the body; for instance, to be over much occupied about exercise, about eating and drinking, about easing oneself, about sexual intercourse. But all these things should be done by the way, and only the mind receive your full attention.

    1ἀφυίας σημεῖον.

    42   WHEN some one may do you an injury or speak ill of you, remember that he does it or speaks it under the impression that it is proper for him to do so. He must then of necessity follow not the appearance which the case presents to you, but that which it presents to him. Wherefore, if it has a bad appearance to him, it is he who is injured, being deceived. For if anyone should take a true statement to be false, it is not the statement which is damaged, but he who is deceived. If then you set out from these principles, you will bear a gentle mind towards anyone who reviles you. For say on each occasion, So it appeared to him.

    43   EVERY matter has two handles; by the one you can carry it, by the other you cannot. If your brother wrongs you, do not take this by the handle He wrongs me, for by that handle you can not carry it; but take it rather by this, He is my brother, nourished with me, and then you will be taking it by a handle by which you can carry it.

    44   THERE is no nexus in the following reasonings: I am richer than you, therefore I am superior to you; I am more eloquent than you, therefore I am superior. But the nexus is rather in these: I am richer than you, therefore my wealth is superior to yours; I am more eloquent than you, therefore my language is superior to yours. But you are not wealth, and you are not language.

    45   DOES a man bathe himself quickly? Then say not Badly, but Quickly. Does be drink much wine? Then say not Too much, but Much. For before you have discerned how things appear to him,1 how can you know if it were done badly? Thus it will not happen to you to surrender yourself to certain of the phantasms which lay hold of the mind, while comprehending others.

    1This is the usual rendering of  πρὶν γὰρ διαγνῶναι τὸ δόγμα.  I am not sure whether it may not mean  ‘Until you have analyzed the nature of opinion.’

    46.1   NEVER proclaim yourself a philosopher, nor talk much among the sensual about the philosophic maxims; but do the things that follow from the maxims. For example, do not discourse, at a feast, upon how one ought to eat, but eat as you ought. For remember that even so Socrates everywhere banished ostentation, so that men used to come to him desiring him to recommend them to teachers in philosophy; and he brought them away and did so, so well did he bear being overlooked.

    46.2   And if, among the sensual, discourse should arise concerning some maxim, do you for the most part keep silence; for there is great risk that you straightway vomit up what you, have not digested. And when some one shall say to you that you know nothing, and you are not offended, then know that the work is begun. And as sheep do not bring their food to the shepherds, to show how much they have eaten, but digesting inwardly their provender bear outwardly wool and milk, even so do not you, for the most part, display your maxims before the sensual, but rather the works which follow from them, when they are digested.

    47   WHEN you have harmonized yourself to a frugal provision for your bodily needs, do not pride yourself on that; and if your drink is water, do not take every opportunity of declaring I am a water drinker. And if you wish at any time to inure yourself to labour and endurance, do it unto yourself and not unto the world, and do not embrace the statues,1 but some time when you are exceedingly thirsty take a draught of cold water into your mouth, and spit it out, and say nothing about it.

    1Philosophers used sometimes to try (and exhibit) their powers of enduring cold by embracing the statues in the market-places in winter time.

    48.1   THE position and character of the sensualist: he never looks to himself for benefit or harm, but always to external things. The position and character of the sage: be looks for benefit or harm only to himself.

    48.2   The tokens of one who is making advance: he blames none; he praises none; he finds fault with none; he accuses none; he never speaks of himself as being somewhat, or as knowing aught; when he is let or hindered in anything he accuses himself;1 if anyone praises him, he laughs at him in his sleeve; if anyone blames him, he makes no defence; he goes about like convalescents, fearing to move the parts that are settling together before they have taken hold.

    1See chapter v. ad fin.

    48.3   He has taken out of himself all desire,2 and has transferred his aversion solely to things contrary to Nature which depend upon ourselves. He attempts nothing, except lightly and indifferently. If he is thought foolish or unlearned, he is not concerned. In one word, he watches himself as he would a treacherous enemy.

    2See chapter 2.2

    49   WHEN someone himself on act of his powers of understanding and expounding the writings Chrysippus,1 then say to yourself, If Chrysippus had not written obscurely, this man would have nothing  whereon to exalt himself. But I, what do I desire? Is it not to learn to understand Nature and to follow her? I inquire, then, who can expound Nature to me and hearing that Chrysippus can, I betake myself to him. But I do not understand his writings, therefore I seek an expounder for them. And so far there is nothing exalted. But when I have found the expounder, it remains for me to put into practice what he declares to me; and in this alone is there anything exalted. But if I look on the exposition as a thing to be admired in itself, what else am I become than a grammarian instead of a sage? except that the exposition is not of Homer but of Chrysippus. Therefore when one may ask me to lecture on the philosophy of Chrysippus, I shall rather blush when I am not able to show forth works of a like nature and in harmony with the words.

    1 Chrysippus ( 3rd century B.C.) united brilliant talents with the true Stoic independence of character, and was very famous in his own day and for long afterwards. He philosophized with a fearless idealism, resolved to follow logic wherever it should lead him. In consequence, perhaps, of hypotheses put forward in this spirit, he was accused by some(εἰσὶν οἵ)  of having advocated cannibalism and incest. None of his many writings are now extant, but Diogenes Laertius gives some traits of him.

    50   WHATSOEVER things are preferred, in these abide as in laws which it were impious to transgress. And whatever anyone may say of you, regard it not, for neither does this concern your self.

    51.1   HOW long will you still delay to hold yourself worthy of the best things, and to transgress in nothing the defining word?1 You have accepted the maxims by which you ought to live, and do you live by them? What teacher do you still look for, to whom to hand over the task of your correction? You are no longer a boy, but now a full grown man.  If, then,you are neglectful and indifferent, and make delay after delay, and form purpose after purpose, and fix again and again the days after which you will begin to attend to yourself, you will not see that you are making no advance, but will be now and always a sensualist, living and dying.

    1Καὶ ἐν μηδενὶ παραβαίνειν τὸν διαιροῦντα λόγον., ‘Et nulla parte violare Rationem quae rerum distinctionem docet’ (Schweighauser). The defining (or dividing) word is that which declares the distinction between the things that really concern us and those which only appear to do so.

    51.2   Therefore hold yourself worthy forthwith to live as a man of full age, and with your foot on the path; and let whatever appears to you as the best be to you as an inviolable law. And when anything is presented to you which involves toil, or pleasure, or reputation or the loss of it, remember that now is the conflict, here are the Olympic games, and you can put them off no longer; and that in a single day and in a single trial ground is to be lost or gained.

    51.3   It was thus that Socrates made himself what he was, on every occasion bringing forward his true self,1 and never having regard to anything else than Reason, And you, though you are not yet Socrates, yet as one who desires to be Socrates so you ought to live.

    1This translation (for the reading ἐπὶ πάντων προςάγων ἑαυτόν) is unprecedented, and will, no doubt, be criticized. Yet it seems to me that this pregnant use of the pronoun is often distinctly visible in Epictetus, and that this translation should not seem farfetched to anyone who has found reason in the view of his teaching which I have tried to suggest in the Preface.

    52.1   THE first and most necessary point in philosophy is the practice of the maxims, for example, not to lie. The second is the proof of the maxims, as, Whence it comes that one ought not to lie. The third is that which gives confirmation and coherence to these, as, Whence it comes that this is a proof, for what is proof? what is consequence? what is contradiction? what is truth? what is falsehood?

    52.2   Thus the third point is necessary through the second, and the second through the first. But the most necessary of all, and that where we must rest, is the first. But we do the contrary. For we delay in the third point, and spend all our zeal upon that, while of the first we are utterly careless: we are liars, while we are ready in explaining how it is shown that it is wrong to lie.

    53.1   HOLD in readiness for every heed, these—

    ‘Lead me, O Zeus, and thou Destiny, whithersoever ye have appointed me to go, and may I follow fearlessly.

    ‘But if in an evil mind I be unwilling, still must I follow,’1

    1This quotation is from the Stoic philosopher Cleanthes(263 B.C.). from whose Hymn to Zeus St. Paul is supposed to have quoted the words, ‘For we are also his offspring’ (Acts xvii. 24).

    53.2       ‘That man is wise among us, and has understanding of things divine, who has nobly agreed with Necessity.’2

    2From a lost play of Euripides.

    53.3   But the third also—

     ‘O Crito, if so it seem good to the Gods, so let it be; Anytus and Melitus are able to kill me if they like, but to harm me, never,’3

     3Epictetus has joined together two sayings of Socrates, one from the Crito, the other from the Apologia. Anytus and Melitus were the two principal accusers of Socrates in the trial which ended in his condemnation to death.
