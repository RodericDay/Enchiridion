On  no  occasion  call yourself  a  philosopher,  nor  talk  at large  of  your
principles among the multitude, but act  on your principles. For instance, at a
banquet do not  say how one ought to  eat, but eat as you  ought. Remember that
Socrates had so completely got rid of the thought of display that when men came
and wanted  an introduction to philosophers  he took them to  be introduced; so
patient of  neglect was he.  And if a discussion  arise among the  multitude on
some principle, keep silent  for the most part; for you are  in great danger of
blurting out some undigested thought. And when  some one says to you, 'You know
nothing', and you do  not let it provoke you, then know that  you are really on
the right road. For  sheep do not bring grass to their  shepherds and show them
how much they have  eaten, but they digest their fodder and  then produce it in
the form  of wool and  milk. Do the same  yourself; instead of  displaying your
principles to the  multitude, show them the results of  the principles you have
digested.
