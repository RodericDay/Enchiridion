The condition and characteristic of a vulgar person is that he never
looks for either help or harm from himself, but only from externals. The
condition and characteristic of a philosopher is that he looks to himself
for all help or harm. The marks of a proficient are that he censures no
one, praises no one, blames no one, accuses no one; says nothing
concerning himself as being anybody or knowing anything. When he is in
any instance hindered or restrained, he accuses himself; and if he is
praised, he smiles to himself at the person who praises him; and if he is
censured, he makes no defense. But he goes about with the caution of a
convalescent, careful of interference with anything that is doing well
but not yet quite secure. He restrains desire; he transfers his aversion
to those things only which thwart the proper use of our own will; he
employs his energies moderately in all directions; if he appears stupid
or ignorant, he does not care; and, in a word, he keeps watch over
himself as over an enemy and one in ambush.
