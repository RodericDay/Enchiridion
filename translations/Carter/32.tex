When you have recourse to divination, remember that you know not what the event
will be, and you come to learn it of  the diviner; but of what nature it is you
know before you come, at least if you are a philosopher. For if it is among the
things not  in our  own control, it  can by  no means be  either good  or evil.
Don't, therefore, bring either desire or aversion with you to the diviner (else
you will approach  him trembling), but first acquire a  distinct knowledge that
every event is indifferent and nothing to  you, of whatever sort it may be, for
it will be in your power to make a right use of it, and this no one can hinder;
then come with confidence to the gods, as your counselors, and afterwards, when
any counsel is given you, remember  what counselors you have assumed, and whose
advice  you will  neglect  if  you disobey.  Come  to  divination, as  Socrates
prescribed, in cases of which the whole consideration relates to the event, and
in which no opportunities are afforded by reason, or any other art, to discover
the thing proposed to be learned. When,  therefore, it is our duty to share the
danger  of a  friend or  of our  country, we  ought not  to consult  the oracle
whether we  will share  it with  them or  not. For,  though the  diviner should
forewarn you  that the victims  are unfavorable, this  means no more  than that
either death or mutilation or exile is portended. But we have reason within us,
and it  directs, even with these  hazards, to the greater  diviner, the Pythian
god, who cast out of the temple the person who gave no assistance to his friend
while another was murdering him.
