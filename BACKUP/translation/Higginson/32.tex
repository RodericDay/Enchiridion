When you have recourse to divination, remember that you know not what the
event will be, and you come to learn it of the diviner; but of what
nature it is you knew before coming; at least, if you are of philosophic
mind. For if it is among the things not within our own power, it can by
no means be either good or evil. Do not, therefore, bring with you to the
diviner either desire or aversion---else you will approach him
trembling---but first clearly understand that every event is indifferent
and nothing to \emph{you}, of whatever sort it may be; for it will be in your
power to make a right use of it, and this no one can hinder. Then come
with confidence to the gods as your counselors; and afterwards, when any
counsel is given you, remember what counselors you have assumed, and
whose advice you will neglect if you disobey. Come to divination as
Socrates prescribed, in cases of which the whole consideration relates to
the event, and in which no opportunities are afforded by reason or any
other art to discover the matter in view. When, therefore, it is our duty
to share the danger of a friend or of our country, we ought not to
consult the oracle as to whether we shall share it with them or not. For
though the diviner should forewarn you that the auspices are unfavorable,
this means no more than that either death or mutilation or exile is
portended. But we have reason within us; and it directs us, even with
these hazards, to stand by our friend and our country. Attend, therefore,
to the greater diviner, the Pythian God, who once cast out of the temple
him who neglected to save his friend.\footnotemark
\footnotetext{This refers to an anecdote given in full by Simplicius, in his
commentary on this passage, of a man assaulted and killed on his way
to consult the oracle, while his companion, deserting him, took refuge
in the temple till cast out by the Deity.---Tr.}
