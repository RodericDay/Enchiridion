Immediately prescribe some character and some form to yourself, which you shall
observe both when you are alone and when you meet with men.

And let silence be the general rule, or let only what is necessary be said, and
in few words.  And rarely and when  the occasion calls we  shall say something;
but about none  of the common subjects, nor about  gladiators, nor horse-races,
nor about athletes, nor about eating or drinking, which are the usual subjects;
and especially  not about men, as  blaming them or praising  them, or comparing
them. If then you are able, bring over by your conversation the conversation of
your  associates to  that which  is  proper; but  if  you should  happen to  be
confined to the company of strangers, be silent.

Let not your laughter be much, nor on many occasions, nor excessive.

Refuse altogether to take  an oath, if it is possible: if it  is not, refuse as
far as you are able.

Avoids banquets  which are given by  strangers and by ignorant  persons. But if
ever there is occasion to join in  them, let your attention be carefully fixed,
that you slip  not into the manners  of the vulgar (the  uninstructed). For you
must know, that if your companion be impure, he also who keeps company with him
must become impure, though he should happen to be pure.

Take (apply)  the things which relate  to the body as  far as the bare  use, as
food, drink, clothing,  house, and slaves: but exclude everything  which is for
show or luxury.

As to pleasure  with women, abstain as  far as you can before  marriage: but if
you do indulge in  it, do it in the way which is  conformable to custom. Do not
however be  disagreeable to those  who indulge  in these pleasures,  or reprove
them; and do not often boast that you do not indulge in them yourself.

If a man has  reported to you, that a certain person speaks  ill of you, do not
make any defense (answer) to what has been told you: but reply, The man did not
know the rest of my faults, for he would not have mentioned these only.

It is not necessary to go to the  theaters often: but if there is ever a proper
occasion for going, do not show yourself  as being a partisan of any man except
yourself, that is, desire only that to be  done which is done, and for him only
to gain the  prize who gains the prize;  for in this way you will  meet with no
hindrance.  But abstain  entirely from  shouts and  laughter at  any (thing  or
person), or  violent emotions. And  when your are come  away, do not  talk much
about what has  passed on the stage,  except about that which may  lead to your
own improvement.  For it is  plain, if  you do talk  much that you  admired the
spectacle (more than you ought).

Do  not go  to  the hearing  of  certain persons'  recitations  nor visit  them
readily. But if  you do attend, observe gravity and  sedateness, and also avoid
making yourself disagreeable.

When you are going  to meet with any person, and particularly  one of those who
are  considered to  be  in a  superior condition,  place  before yourself  what
Socrates or Zeno  would have done in  such circumstances, and you  will have no
difficulty in making a proper use of the occasion.

When  you are  going to  any of  those  who are  in great  power, place  before
yourself that  you will not find  the man at  home, that you will  be excluded,
that the door will not be opened to  you, that the man will not care about you.
And if with all this it is your duty to visit him, bear what happens, and never
say to yourself that it was not worth the trouble. For this is silly, and marks
the character of a man who is offended by externals.

In company take care  not to speak much and excessively about  your own acts or
dangers: for as  it is pleasant to you  to make mention of your  dangers, it is
not so pleasant to others to hear what  has happened to you. Take care also not
to provoke laughter;  for this is a  slippery way toward vulgar  habits, and is
also adapted to diminish the respect of your neighbors. It is a dangerous habit
also to  approach obscene  talk. When  then anything of  this kind  happens, if
there is a good opportunity, rebuke the man who has proceeded to this talk: but
if there  is not  an opportunity, by  your silence at  least, and  blushing and
expression of  dissatisfaction by your  countenance, show plainly that  you are
displeased at such talk.
