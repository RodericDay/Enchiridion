The first and most necessary topic in philosophy is that of the
use of moral theorems, such as, ``We ought not to lie;'' the second
is that of demonstrations, such as, ``What is the origin of our obligation
not to lie;'' the third gives strength and articulation to the other
two, such as, ``What is the origin of this is a demonstration.'' For
what is demonstration? What is consequence? What contradiction? What
truth? What falsehood? The third topic, then, is necessary on the
account of the second, and the second on the account of the first.
But the most necessary, and that whereon we ought to rest, is the
first. But we act just on the contrary. For we spend all our time
on the third topic, and employ all our diligence about that, and entirely
neglect the first. Therefore, at the same time that we lie, we are
immediately prepared to show how it is demonstrated that lying is
not right. 
