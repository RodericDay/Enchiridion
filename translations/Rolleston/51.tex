The first and most necessary point in philosophy is the practice of the maxims,
for example, not to  lie. The second is the proof of the  maxims, as, whence it
comes that one ought not to lie. The third is that which gives confirmation and
coherence to  these, as,  whence it  comes that this  is a  proof, for  what is
proof?  What is  consequence? What  is contradiction?  What is  truth? What  is
falsehood?

Thus the  third point is necessary  through the second, and  the second through
the first. But the  most necessary of all, and that where we  must rest, is the
first. But we do  the contrary. For we delay in the third  point, and spend all
our zeal upon that,  while of the first we are utterly  careless: we are liars,
while we are ready in explaining how it is shown that it is wrong to lie.
