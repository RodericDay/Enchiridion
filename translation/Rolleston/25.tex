Is some  one preferred  before you  at a feast  or in  salutation, or  in being
invited to give counsel? Then if these  things be good, you should rejoice that
he has  gained them; but if  evil, why grieve  that you have not?  But remember
that if you do not do as other men, in order to gain the things that depend not
upon ourselves, neither shall you be rewarded as they.

For how is it possible for you to  have an equal share with him who hangs about
other men's doors, and attends upon  them, and flatters them, when you yourself
will do none of these things? You  are unjust then, and insatiable, if you wish
to gain  the things that  depend not upon  ourselves, for nothing,  and without
paying the price for which they are sold.

But how much is your head of lettuce  sold for? A penny perchance. Go to, then:
if one will lay out  a penny he may have a head of lettuce;  but you who do not
choose to  lay out your  penny shall  not have your  lettuce. But you  must not
suppose that you will  be therefore worse off than he. For  he has the lettuce,
but you the penny which you did not choose to part with.

And in this matter  also the same principle holds good. You  are not invited to
somebody's banquet? That is because you  did not give the entertainer the price
that banquets are sold for---and they are  sold for flattery, they are sold for
attendance. Pay then  the price if you  think you will profit  by the exchange.
But if you are determined not to lay  out these things, and at the same time to
gain the others---surely you are a greedy man, and an infatuated.

Shall you  have nothing  then instead  of the  banquet which  you give  up? Yea
verily, you shall have this---not to have  praised one whom you did not care to
praise, nor to have endured the insolence of a rich man's doorkeepers.
