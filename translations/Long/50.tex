Whatever things (rules) are proposed to you  (for the conduct of life) abide by
them,  as if  they were  laws, as  if you  would be  guilty of  impiety if  you
transgressed any  of them.  And whatever any  man shall say  about you,  do not
attend  to  it: for  this  is  no  affair of  yours.  How  long will  you  then
still  defer thinking  yourself worthy  of the  best things,  and in  no matter
transgressing the distinctive  reason? Have you accepted  the theorems (rules),
which it was your  duty to agree to, and have you agreed  to them? what teacher
then do you still expect that you  defer to him the correction of yourself? You
are no longer a youth, but already a  full grown man. If then you are negligent
and slothful, and are continually making procrastination after procrastination,
and proposal (intention) after proposal, and  fixing day after day, after which
you  will attend  to  yourself, you  will  not  know that  you  are not  making
improvement, by you  will continue ignorant (uninstructed) both  while you live
and till you die. Immediately then think  it right to live as a full-grown man,
and one who is making proficiency, and  let every thing which appears to you to
be the  best be to you  a law which must  not be transgressed. And  if anything
laborious, or pleasant or glorious or  inglorious be presented to you, remember
that  now is  the  contest, now  are  the  Olympic games,  and  they cannot  be
deferred; and that it depends on one defeat and one giving way that progress is
either lost or  maintained. Socrates in this way became  perfect, on all things
improving himself, attending  to nothing except to reason. But  you, though you
are not yet Socrates, ought to live as one who wishes to be a Socrates.
