Whatever moral rules you have deliberately proposed to yourself.
abide by them as they were laws, and as if you would be guilty of
impiety by violating any of them. Don't regard what anyone says of
you, for this, after all, is no concern of yours. How long, then,
will you put off thinking yourself worthy of the highest improvements
and follow the distinctions of reason? You have received the philosophical
theorems, with which you ought to be familiar, and you have been familiar
with them. What other master, then, do you wait for, to throw upon
that the delay of reforming yourself? You are no longer a boy, but
a grown man. If, therefore, you will be negligent and slothful, and
always add procrastination to procrastination, purpose to purpose,
and fix day after day in which you will attend to yourself, you will
insensibly continue without proficiency, and, living and dying, persevere
in being one of the vulgar. This instant, then, think yourself worthy
of living as a man grown up, and a proficient. Let whatever appears
to be the best be to you an inviolable law. And if any instance of
pain or pleasure, or glory or disgrace, is set before you, remember
that now is the combat, now the Olympiad comes on, nor can it be put
off. By once being defeated and giving way, proficiency is lost, or
by the contrary preserved. Thus Socrates became perfect, improving
himself by everything. attending to nothing but reason. And though
you are not yet a Socrates, you ought, however, to live as one desirous
of becoming a Socrates. 
