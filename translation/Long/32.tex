When you have recourse to divination, remember that you do not know how it will
turn out, but that  you are come to inquire from the diviner.  But of what kind
it is, you  know when you come, if  indeed you are a philosopher. For  if it is
any of the things  which are not in our power, it  is absolutely necessary that
it must  be neither good nor  bad. Do not then  bring to the diviner  desire or
aversion: if you do, you will approach  him with fear. But having determined in
your mind  that everything which  shall turn  out (result) is  indifferent, and
does not concern you,  and whatever it may be, for it will  be in your power to
use it well, and no man will hinder this, come then with confidence to the Gods
as your advisers. And then when any advice shall have been given, remember whom
you have  taken as advisers, and  whom you will  have neglected, if you  do not
obey them. And go  to divination, as Socrates said that  you ought, about those
matters in  which all  the inquiry has  reference to the  result, and  in which
means are not given either by reason nor by any other art for knowing the thing
which  is the  subject of  the  inquiry. Wherefore  when  we ought  to share  a
friend's  danger or  that of  our  country, you  must not  consult the  diviner
whether you ought to share it. For even  if the diviner shall tell you that the
signs of the victims are unlucky, it is  plain that this is a token of death or
mutilation of part of the body or  of exile. But reason prevails that even with
these risks  we should  share the  dangers of  our friend  and of  our country.
Therefore attend to the greater diviner,  the Pythian God, who ejected from the
temple him who did not assist his friend when he was being murdered.
