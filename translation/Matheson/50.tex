Whatever principles you put before you, hold fast to them as laws which it will
be impious to transgress. But pay no heed to what any one says of you; for this
is something beyond your own control.

How long will you  wait to think yourself worthy of  the highest and transgress
in nothing  the clear pronouncement of  reason? You have received  the precepts
which you ought  to accept, and you  have accepted them. Why then  do you still
wait for a master, that you may  delay the amendment of yourself till he comes?
You  are a  youth no  longer, you  are now  a full-grown  man. If  now you  are
careless and indolent and are always  putting off, fixing one day after another
as the  limit when  you mean to  begin attending to  yourself, then,  living or
dying,  you will  make no  progress but  will continue  unawares in  ignorance.
Therefore make up your mind before it is  too late to live as one who is mature
and proficient,  and let all that  seems best to you  be a law that  you cannot
transgress. And if  you encounter anything troublesome or  pleasant or glorious
or inglorious, remember that the hour  of struggle is come, the Olympic contest
is  here and  you may  put  off no  longer, and  that  one day  and one  action
determines whether the progress you have achieved is lost or maintained.

This was how  Socrates attained perfection, paying heed to  nothing but reason,
in all that he  encountered. And if you are not yet Socrates,  yet ought you to
live as one who would wish to be a Socrates.
