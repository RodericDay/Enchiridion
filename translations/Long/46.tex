On no  occasion call yourself  a philosopher, and do  not speak much  among the
uninstructed about theorems (philosophical rules,  precepts): but do that which
follows from them. For example at a banquet  do not say how a man ought to eat,
but  eat as  you ought  to eat.  For remember  that in  this way  Socrates also
altogether  avoided ostentation:  persons used  to come  to him  and ask  to be
recommended by him  to philosophers, and he used to  take them to philosophers:
so easily  did he submit to  being overlooked. Accordingly if  any conversation
should arise among uninstructed persons about any theorem, generally be silent;
for there is great danger that you  will immediately vomit up what you have not
digested. And when a  man shall say to you, that you know  nothing, and you are
not vexed, then be sure that you  have begun the work (of philosophy). For even
sheep do not vomit up their grass and  show to the shepherds how much they have
eaten;  but  when they  have  internally  digested  the pasture,  they  produce
externally  wool  and  milk.  Do  you  also  show  not  your  theorems  to  the
uninstructed, but show the acts which come from their digestion.
