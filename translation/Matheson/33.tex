Lay down  for yourself from  the first a definite  stamp and style  of conduct,
which you will maintain  when you are alone and also in the  society of men. Be
silent for the most part, or, if you speak, say only what is necessary and in a
few  words. Talk,  but  rarely, if  occasion  calls  you, but  do  not talk  of
ordinary things---of  gladiators, or horse-races,  or athletes, or of  meats or
drinks---these are  topics that  arise everywhere---but above  all do  not talk
about  men  in  blame  or  compliment  or comparison.  If  you  can,  turn  the
conversation of your company  by your talk to some fitting  subject; but if you
should chance to be isolated among strangers, be silent. Do not laugh much, nor
at many things, nor without restraint.

Refuse to  take oaths, altogether if  that be possible,  but if not, as  far as
circumstances allow.

Refuse the entertainments of strangers and the vulgar. But if occasion arise to
accept them,  then strain every  nerve to avoid lapsing  into the state  of the
vulgar. For know that, if your comrade  have a stain on him, he that associates
with him must needs share the stain, even though he be clean in himself.

For your  body take  just so  much as your  bare need  requires, such  as food,
drink, clothing,  house, servants, but  cut down all  that tends to  luxury and
outward show.

Avoid impurity to the utmost of your  power before marriage, and if you indulge
your passion, let it be done lawfully. But do not be offensive or censorious to
those who indulge  it, and do not  be always bringing up your  own chastity. If
some one  tells you that so  and so speaks ill  of you, do not  defend yourself
against what  he says, but  answer, ``He  did not know  my other faults,  or he
would not have mentioned these alone.''

It is  not necessary for the  most part to go  to the games; but  if you should
have occasion  to go, show  that your first concern  is for yourself;  that is,
wish that only to  happen which does happen, and him only to  win who does win,
for so  you will suffer  no hindrance. But  refrain entirely from  applause, or
ridicule, or  prolonged excitement. And  when you go away  do not talk  much of
what happened there, except so far as it tends to your improvement. For to talk
about it implies that the spectacle excited your wonder.

Do not go lightly or casually to hear lectures; but if you do go, maintain your
gravity and dignity and  do not make yourself offensive. When  you are going to
meet any  one, and particularly some  man of reputed eminence,  set before your
mind the thought, ``What  would Socrates or Zeno have done?''  and you will not
fail to make proper  use of the occasion. When you go to  visit some great man,
prepare your mind by  thinking that you will not find him in,  that you will be
shut out, that the doors will be slammed in your face, that he will pay no heed
to you. And if in  spite of all this you find it fitting for  you to go, go and
bear what happens and never say to yourself, ``It was not worth all this;'' for
that shows a vulgar mind and one at odds with outward things.

In your  conversation avoid frequent  and disproportionate mention of  your own
doings or adventures; for other people do not take the same pleasure in hearing
what has happened to you as you take in recounting your adventures.

Avoid  raising  men's laughter;  for  it  is a  habit  that  easily slips  into
vulgarity, and it may well suffice to lessen your neighbour's respect.

It is  dangerous too  to lapse into  foul language; when  anything of  the kind
occurs, rebuke the offender,  if the occasion allow, and if  not, make it plain
to him by your silence, or a blush or a frown, that you are angry at his words.
