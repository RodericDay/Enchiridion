The  first  and  most  necessary   department  of  philosophy  deals  with  the
application of principles; for instance, ``not  to lie.'' The second deals with
demonstrations; for instance,  ``How comes it that one ought  not to lie?'' The
third  is  concerned  with  establishing and  analysing  these  processes;  for
instance, ``How comes  it that this is a demonstration?  What is demonstration,
what is consequence,  what is contradiction, what is true,  what is false?'' It
follows then that the third department  is necessary because of the second, and
the second because of the first. The first is the most necessary part, and that
in which we must  rest. But we reverse the order: we  occupy ourselves with the
third, and  make that our whole  concern, and the first  we completely neglect.
Wherefore we  lie, but are  ready enough with  the demonstration that  lying is
wrong.
