Appropriate acts  are in general measured  by the relations they  are concerned
with. ``He is your father.'' This means you  are called on to take care of him,
give way to him in all things, bear with him if he reviles or strikes you.

``But he is a bad father.''

Well, have you any natural claim to a good father? No, only to a father.

``My brother wrongs me.''

Be careful then to  maintain the relation you hold to him,  and do not consider
what he does,  but what you must do  if your purpose is to keep  in accord with
nature. For  no one  shall harm  you, without  your consent;  you will  only be
harmed, when you think you are harmed. You will only discover what is proper to
expect  from neighbour,  citizen, or  praetor,  if you  get into  the habit  of
looking at the relations implied by each.
