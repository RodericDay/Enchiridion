The first and most necessary place (part)  in philosophy is the use of theorems
(precepts), for  instance, that  we must not  lie: the second  part is  that of
demonstrations, for instance,  How is it proved  that we ought not  to lie: the
third is that which is confirmatory  of these two and explanatory, for example,
How is  this a demonstration? For  what is demonstration, what  is consequence,
what is contradiction, what is truth, what is falsehood? The third part (topic)
is necessary on account of the second,  and the second on account of the first;
but the most necessary and that on which  we ought to rest is the first. But we
do  the contrary.  For  we spend  our  time on  the third  topic,  and all  our
earnestness is about  it: but we entirely neglect the  first. Therefore we lie;
but the demonstration that we ought not to lie we have ready to hand.
