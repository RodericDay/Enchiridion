If religion towards the Gods, know that  the principal element is to have right
opinions about them, as  existing, and as managing the Whole  in fair order and
justice; and  then to set yourself  to obey them, and  to yield to them  in all
that happens  and to  follow it  willingly, as  accomplished under  the highest
counsels. For thus you  will never at any time blame the  Gods, nor accuse them
of being neglectful.

But this  can be  done in no  other way than  by placing  good and evil  in the
things that depend upon ourselves, and withdrawing them from those that do not;
for if you take  any of the latter things for good or  evil, then when you fail
to obtain what you desire, and fall into what you would avoid, it is inevitable
that you must blame and hate those who caused you to do so.

For it is the inherent nature of every creature to fly from and shun all things
that appear harmful, and the causes of the same, but to follow after and admire
things, and  the causes of  things, which  appear beneficial. It  is impossible
then that one who thinks himself harmed should take delight in that which seems
to harm him, just  as it is impossible that he should take  delight in the very
injury itself.

And thus it is that a father is reviled  by his son when he will not give him a
share of  the things  that are thought  to be  good. And it  was this  that set
Polyneices  and Eteocles  at war  with each  other, the  opinion, namely,  that
royalty was a good. And through this the Gods are blamed by the husband man and
the sailor, by the merchant and by  those who lose their wives or children. For
when advantage is, there also is religion.  So that he who is careful to desire
and to dislike as he ought is careful at the same time of religion.

But it is right too that every man should pour libations, and offer sacrifices,
and offer first-fruits according to the customs of his fathers, purely, and not
supinely, nor negligently, nor indeed scantily, yet not beyond his means.
