Remember that desire demands the attainment  of that of which you are desirous;
and aversion demands the avoidance of that to which you are averse; that he who
fails of  the object  of his  desires is  disappointed; and  he who  incurs the
object of his  aversion is wretched. If, then, you  shun only those undesirable
things which you can control, you will never incur anything which you shun; but
if  you  shun  sickness, or  death,  or  poverty,  you  will run  the  risk  of
wretchedness. Remove  [the habit of] aversion,  then, from all things  that are
not within our power,  and apply it to things undesirable  which are within our
power. But for  the present, altogether restrain desire; for  if you desire any
of the things  not within our own power, you  must necessarily be disappointed;
and you  are not yet  secure of those  which are within  our power, and  so are
legitimate objects  of desire.  Where it  is practically  necessary for  you to
pursue  or avoid  anything, do  even this  with discretion  and gentleness  and
moderation.
