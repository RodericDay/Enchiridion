Then you  go to  inquire of  an oracle, remember  that though  you do  not know
beforehand what the event will be (for this very thing is what you have come to
learn from the seer), yet of what nature it must be (if you are a philosopher),
you knew  already when  you came. For  if it  be of those  things which  do not
depend upon ourselves, it follows of necessity  that it can be neither good nor
evil.

When you go then  to the seer, bring with you neither  desire nor aversion, and
approach him, not  with trembling, but with the full  assurance that all events
are indifferent, and nothing  to you, and that whatever may  befall you it will
be your part to use it nobly; and this  no one can prevent. Go then with a good
courage to  the Gods as  to counselors, and, for  the rest, when  something has
been  counseled  to  you, remember  who  they  are  whom  you have  chosen  for
counselors, and whom you will be slighting if you are not obedient.

Therefore, as Socrates  insisted, you should consult the oracle  in those cases
only where your judgment depends entirely upon the event, and where you have no
resources, either from reason or any  other method, for knowing beforehand what
is independently  certain in the  case. Thus, when it  may behove you  to share
some danger with  your friend or your  country, do not inquire  whether you may
[safely] do so. For if the seer  should announce to you that the sacrifices are
inauspicious,  that clearly  signifies  death, or  the loss  of  some limb,  or
banishment; yet Reason  convinces that even with these things  you should stand
by your  friend and share  your country's  danger. Mark therefore  that greater
seer, the  Pythian, who cast  out of  the temple one  who, when his  friend was
being murdered, did not help him.
