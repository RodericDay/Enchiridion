In every affair consider what precedes and what follows, and then
undertake it. Otherwise you will begin with spirit, indeed, careless of
the consequences, and when these are developed, you will shamefully
desist. ``I would conquer at the Olympic Games.'' But consider what
precedes and what follows, and then, if it be for your advantage, engage
in the affair. You must conform to rules, submit to a diet, refrain from
dainties; exercise your body, whether you choose it or not, at a stated
hour, in heat and cold; you must drink no cold water, and sometimes no
wine---in a word, you must give yourself up to your trainer as to a
physician. Then, in the combat, you may be thrown into a ditch, dislocate
your arm, turn your ankle, swallow an abundance of dust, receive stripes
[for negligence], and, after all, lose the victory. When you have
reckoned up all this, if your inclination still holds, set about the
combat. Otherwise, take notice, you will behave like children who
sometimes play wrestlers, sometimes gladiators, sometimes blow a trumpet,
and sometimes act a tragedy, when they happen to have seen and admired
these shows. Thus you too will be at one time a wrestler, and another a
gladiator; now a philosopher, now an orator; but nothing in earnest. Like
an ape you mimic all you see, and one thing after another is sure to
please you, but is out of favor as soon as it becomes familiar. For you
have never entered upon anything considerately; nor after having surveyed
and tested the whole matter, but carelessly, and with a halfway zeal.
Thus some, when they have seen a philosopher and heard a man speaking
like Euphrates\footnote{Euphrates was a philosopher of Syria, whose character is described,
with the highest encomiums, by Pliny the Younger, \emph{Letters} I. 10.}---though, indeed, who can speak like him?---have a mind to
be philosophers, too. Consider first, man, what the matter is, and what
your own nature is able to bear. If you would be a wrestler, consider
your shoulders, your back, your thighs; for different persons are made
for different things. Do you think that you can act as you do and be a
philosopher, that you can eat, drink, be angry, be discontented, as you
are now? You must watch, you must labor, you must get the better of
certain appetites, must quit your acquaintances, be despised by your
servant, be laughed at by those you meet; come off worse than others in
everything---in offices, in honors, before tribunals. When you have fully
considered all these things, approach, if you please---that is, if, by
parting with them, you have a mind to purchase serenity, freedom, and
tranquillity. If not, do not come hither; do not, like children, be now a
philosopher, then a publican, then an orator, and then one of Caesar's
officers. These things are not consistent. You must be one man, either
good or bad. You must cultivate either your own reason or else externals;
apply yourself either to things within or without you---that is, be either
a philosopher or one of the mob.\footnote{Chapter XV of the third book of the \emph{Discourses}, which, with the
exception of some very trifling differences, is the same as chapter
XXIX of the \emph{Enchiridion}.---Ed.}

