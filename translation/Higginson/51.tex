The first and most necessary topic in philosophy is the practical
application of principles, as, \emph{We ought not to lie}; the second is that
of demonstrations as, \emph{Why it is that we ought not to lie}; the third,
that which gives strength and logical connection to the other two, as,
\emph{Why this is a demonstration}. For what is demonstration? What is a
consequence? What a contradiction? What truth? What falsehood? The third
point is then necessary on account of the second; and the second on
account of the first. But the most necessary, and that whereon we ought
to rest, is the first. But we do just the contrary. For we spend all our
time on the third point and employ all our diligence about that, and
entirely neglect the first. Therefore, at the same time that we lie, we
are very ready to show how it is demonstrated that lying is wrong.
