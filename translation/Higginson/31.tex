Be assured that the essence of piety toward the gods lies in this---to form
right opinions concerning them, as existing and as governing the universe
justly and well. And fix yourself in this resolution, to obey them, and
yield to them, and willingly follow them amidst all events, as being
ruled by the most perfect wisdom. For thus you will never find fault with
the gods, nor accuse them of neglecting you. And it is not possible for
this to be affected in any other way than by withdrawing yourself from
things which are not within our own power, and by making good or evil to
consist only in those which are. For if you suppose any other things to
be either good or evil, it is inevitable that, when you are disappointed
of what you wish or incur what you would avoid, you should reproach and
blame their authors. For every creature is naturally formed to flee and
abhor things that appear hurtful and that which causes them; and to
pursue and admire those which appear beneficial and that which causes
them. It is impracticable, then, that one who supposes himself to be hurt
should rejoice in the person who, as he thinks, hurts him, just as it is
impossible to rejoice in the hurt itself. Hence, also, a father is
reviled by his son when he does not impart the things which seem to be
good; and this made Polynices and Eteocles\footnotemark mutually enemies---that
empire seemed good to both. On this account the husbandman reviles the
gods; [and so do] the sailor, the merchant, or those who have lost wife
or child. For where our interest is, there, too, is piety directed. So
that whoever is careful to regulate his desires and aversions as he ought
is thus made careful of piety likewise. But it also becomes incumbent on
everyone to offer libations and sacrifices and first fruits, according to
the customs of his country, purely, and not heedlessly nor negligently;
not avariciously, nor yet extravagantly.
\footnotetext{The two inimical sons of Oedipus, who killed each other in
 battle.---Ed.}
