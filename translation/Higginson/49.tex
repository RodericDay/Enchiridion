When anyone  shows himself vain on  being able to understand  and interpret the
works  of Chrysippus,\footnotemark  say  to yourself:  ``Unless Chrysippus  had
written obscurely, this person  would have had nothing to be  vain of. But what
do I desire? To understand nature, and  follow her. I ask, then, who interprets
her;  and hearing  that Chrysippus  does,  I have  recourse  to him.  I do  not
understand his  writings. I seek,  therefore, one  to interpret them.''  So far
there is  nothing to value  myself upon. And when  I find an  interpreter, what
remains is to make  use of his instructions. This alone  is the valuable thing.
But  if I  admire merely  the  interpretation, what  do  I become  more than  a
grammarian, instead of  a philosopher, except, indeed, that instead  of Homer I
interpret Chrysippus? When anyone, therefore,  desires me to read Chrysippus to
him,  I rather  blush when  I cannot  exhibit actions  that are  harmonious and
consonant with his discourse. \footnotetext{Chrysippus (c. 280--207 B.C.) was a
Stoic  philosopher who  became head  of the  Stoa after  Cleanthes. His  works,
which  are lost,  were  most influential  and were  generally  accepted as  the
authoritative interpretation of orthodox Stoic philosophy.---Ed.}
