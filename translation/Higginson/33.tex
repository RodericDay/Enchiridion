Begin by prescribing  to yourself some character and demeanor,  such as you may
preserve both alone and in company.

Be mostly silent,  or speak merely what  is needful, and in few  words. We may,
however, enter sparingly into discourse  sometimes, when occasion calls for it;
but let  it not  run on  any of the  common subjects,  as gladiators,  or horse
races,  or  athletic  champions,  or  food, or  drink---the  vulgar  topics  of
conversation---and especially not on men, so  as either to blame, or praise, or
make comparisons. If  you are able, then, by your  own conversation, bring over
that of  your company to  proper subjects; but if  you happen to  find yourself
among strangers, be silent.

Let not your laughter be loud, frequent, or abundant.

Avoid taking  oaths, if possible,  altogether; at any rate,  so far as  you are
able.

Avoid public  and vulgar entertainments; but  if ever an occasion  calls you to
them, keep  your attention  upon the  stretch, that  you may  not imperceptibly
slide into vulgarity. For be assured that  if a person be ever so pure himself,
yet, if his companion be corrupted, he who converses with him will be corrupted
likewise.

Provide things relating to the body  no further than absolute need requires, as
meat, drink, clothing, house, retinue. But cut off everything that looks toward
show and luxury.

Before marriage guard yourself with  all your ability from unlawful intercourse
with women; yet be  not uncharitable or severe to those who  are led into this,
nor boast frequently that you yourself do otherwise.

If  anyone tells  you that  a certain  person speaks  ill of  you, do  not make
excuses about what  is said of you,  but answer: ``He was ignorant  of my other
faults, else he would not have mentioned these alone.''

It is not necessary  for you to appear often at public  spectacles; but if ever
there is a proper  occasion for you to be there, do  not appear more solicitous
for any other than for yourself---that is,  wish things to be only just as they
are, and only  the best man to win;  for thus nothing will go  against you. But
abstain entirely from acclamations and  derision and violent emotions. And when
you  come away,  do not  discourse a  great deal  on what  has passed  and what
contributes  nothing  to your  own  amendment.  For  it  would appear  by  such
discourse that you were dazzled by the show.

Be not  prompt or ready  to attend private recitations;  but if you  do attend,
preserve your gravity and dignity, and yet avoid making yourself disagreeable.

When you  are going to  confer with anyone, and  especially with one  who seems
your superior, represent to yourself how Socrates or Zeno\footnote{Reference is
to Zeno of Cyprus (335-263 B.C.), the founder of the Stoic school.---Ed.} would
behave in such a case, and you will  not be at a loss to meet properly whatever
may occur.

When you are going  before anyone in power, fancy to yourself  that you may not
find him at home, that you may be shut out, that the doors may not be opened to
you, that he may not notice you. If, with all this, it be your duty to go, bear
what happens and never say to yourself,  ``It was not worth so much''; for this
is vulgar, and like a man bewildered by externals.

In company,  avoid a  frequent and  excessive mention of  your own  actions and
dangers. For however agreeable it may be to yourself to allude to the risks you
have run, it is not equally agreeable  to others to hear your adventures. Avoid
likewise  an endeavor  to  excite  laughter, for  this  may  readily slide  you
into  vulgarity, and,  besides,  may be  apt  to  lower you  in  the esteem  of
your  acquaintance. Approaches  to indecent  discourse are  likewise dangerous.
Therefore, when anything of this sort happens, use the first fit opportunity to
rebuke him who makes  advances that way, or, at least,  by silence and blushing
and a serious look show yourself to be displeased by such talk.
