When some one may  do you an injury or speak ill of  you, remember that he does
it or  speaks it under the  impression that it is  proper for him to  do so. He
must then  of necessity follow  not the appearance  which the case  presents to
you, but that which  it presents to him. Wherefore, if it  has a bad appearance
to him, it  is he who is injured,  being deceived. For if anyone  should take a
true statement to  be false, it is  not the statement which is  damaged, but he
who is  deceived. If then you  set out from  these principles, you will  bear a
gentle mind towards anyone who reviles  you. For say on each occasion, \emph{So
it appeared to him.}
