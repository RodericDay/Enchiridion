When you make use  of prophecy remember that while you know  not what the issue
will be, but are come to learn it from the prophet, you do know before you come
what manner of thing  it is, if you are really a philosopher.  For if the event
is not in our control, it cannot be either good or evil. Therefore do not bring
with you  to the  prophet the  will to  get or the  will to  avoid, and  do not
approach him with trembling,  but with your mind made up,  that the whole issue
is indifferent and does not affect you and  that, whatever it be, it will be in
your  power to  make  good  use of  it,  and no  one  shall  hinder this.  With
confidence then approach the gods as counsellors, and further, when the counsel
is given you, remember  whose counsel it is, and whom  you will be disregarding
if you  disobey. And consult the  oracle, as Socrates thought  men should, only
when the whole question turns upon the  issue of events, and neither reason nor
any art  of man provides  opportunities for  discovering what lies  before you.
Therefore, when it  is your duty to  risk your life with friend  or country, do
not ask the oracle whether you should  risk your life. For if the prophet warns
you that  the sacrifice  is unfavourable,  though it is  plain that  this means
death or exile  or injury to some  part of your body, yet  reason requires that
even  at this  cost you  must stand  by your  friend and  share your  country's
danger. Wherefore pay heed to the greater prophet, Pythian Apollo, who cast out
of his temple the man who did not help his friend when he was being killed.
