Remember that  desire announces  the aim  of attaining  the thing  desired, and
aversion that of not falling into the  thing shunned; and that to miss what you
desire is unfortunate, but it is misfortune to fall into what you shun, But you
can  never fall  into anything  that you  shun, if  you will  shun only  things
contrary to  Nature which lie  within your power: but  if you shun  disease, or
death, or poverty,  you will have misfortune, Withdraw then  your aversion from
those things that do not depend upon  ourselves, and place it upon those things
contrary to Nature which do depend upon ourselves.

And let desire,  for the present, be  utterly effaced; for if  you are desiring
something of  the kind that  does not depend upon  ourselves, it must  needs be
that you miscarry, and of the things  that do depend upon ourselves, of such as
you may fairly desire, none are yet open to you. Use therefore only [tentative]
advances  and  withdrawals, and  that  but  lightly,  and with  exception,  and
indifferently.
