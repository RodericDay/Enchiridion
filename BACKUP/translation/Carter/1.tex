Some things are in our control and others not. Things in our control
are opinion, pursuit, desire, aversion, and, in a word, whatever are
our own actions. Things not in our control are body, property, reputation,
command, and, in one word, whatever are not our own actions.

The things in our control are by nature free, unrestrained, unhindered;
but those not in our control are weak, slavish, restrained, belonging
to others. Remember, then, that if you suppose that things which are
slavish by nature are also free, and that what belongs to others is
your own, then you will be hindered. You will lament, you will be
disturbed, and you will find fault both with gods and men. But if
you suppose that only to be your own which is your own, and what belongs
to others such as it really is, then no one will ever compel you or
restrain you. Further, you will find fault with no one or accuse no
one. You will do nothing against your will. No one will hurt you,
you will have no enemies, and you not be harmed. 

Aiming therefore at such great things, remember that you must not
allow yourself to be carried, even with a slight tendency, towards
the attainment of lesser things. Instead, you must entirely quit some
things and for the present postpone the rest. But if you would both
have these great things, along with power and riches, then you will
not gain even the latter, because you aim at the former too: but you
will absolutely fail of the former, by which alone happiness and freedom
are achieved. 

Work, therefore to be able to say to every harsh appearance, ``You
are but an appearance, and not absolutely the thing you appear to
be.'' And then examine it by those rules which you have, and first,
and chiefly, by this: whether it concerns the things which are in
our own control, or those which are not; and, if it concerns anything
not in our control, be prepared to say that it is nothing to you.
