Remember that  desire contains in  it the  profession (hope) of  obtaining that
which you desire; and the profession  (hope) in aversion (turning from a thing)
is that  you will not  fall into that  which you attempt  to avoid: and  he who
fails in his desire  is unfortunate; and he who falls into  that which he would
avoid, is  unhappy. If then  you attempt to avoid  only the things  contrary to
nature which  are within your  power, you  will not be  involved in any  of the
things which you would  avoid. But if you attempt to avoid  disease or death or
poverty, you will be unhappy. Take away then aversion from all things which are
not in our power, and transfer it to the things contrary to nature which are in
our power.  But destroy desire  completely for the  present. For if  you desire
anything which is not in our power,  you must be unfortunate: but of the things
in our power, and which it would be  good to desire, nothing yet is before you.
But employ only the power of moving  toward an object and retiring from it; and
these powers indeed only slightly and with exceptions and with remission.
