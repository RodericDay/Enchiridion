For piety towards the gods know that  the most important thing is this: to have
right opinions about them—that they exist,  and that they govern the universe
well and justly—and to have set yourself to obey them, and to give way to all
that happens, following  events with a free  will, in the belief  that they are
fulfilled by  the highest  mind. For thus  you will never  blame the  gods, nor
accuse them  of neglecting you. But  this you cannot achieve,  unless you apply
your conception of good  and evil to those things only which  are in our power,
and not to those which are out of our

p. 477

power. For  if you apply your  notion of good or  evil to the latter,  then, as
soon as you fail to get what you will  to get or fail to avoid what you will to
avoid, you  will be  bound to blame  and hate those  you hold  responsible. For
every  living creature  has a  natural tendency  to avoid  and shun  what seems
harmful and all  that causes it, and  to pursue and admire what  is helpful and
all that causes it. It is not possible  then for one who thinks he is harmed to
take pleasure  in what he thinks  is the author of  the harm, any more  than to
take pleasure in the  harm itself. That is why a father is  reviled by his son,
when he does not  give his son a share of what the  son regards as good things;
thus Polynices  and Eteocles were  set at enmity  with one another  by thinking
that a king's throne was a good thing.  That is why the farmer, and the sailor,
and the  merchant, and  those who lose  wife or children  revile the  gods. For
men's religion is bound  up with their interest. Therefore he  who makes it his
concern rightly  to direct his will  to get and  his will to avoid,  is thereby
making piety his  concern. But it is  proper on each occasion  to make libation
and sacrifice and to offer first-fruits according to the custom of our fathers,
with  purity and  not in  slovenly or  careless fashion,  without meanness  and
without extravagance.
