In every act  observe the things which  come first, and those  which follow it;
and so proceed  to the act. If you  do not, at first you will  approach it with
alacrity,  without  having  thought  of  the  things  which  will  follow;  but
afterward, when certain  base (ugly) things have shown themselves,  you will be
ashamed. A man wishes to conquer at  the Olympic games. I also wish indeed, for
it is  a fine  thing. But  observe both the  things which  come first,  and the
things which follow;  and then begin the act. You  must do everything according
to  rule, eat  according to  strict orders,  abstain from  delicacies, exercise
yourself as  you are bid  at appointed  times, in heat,  in cold, you  must not
drink cold water, nor wine as you  choose; in a word, you must deliver yourself
up to the exercise  master as you do to the physician, and  then proceed to the
contest. And sometimes  you will strain the  hand, put the ankle  out of joint,
swallow much dust,  sometimes be flogged, and after all  this be defeated. When
you have considered all this, if you still choose, go to the contest: if you do
not, you will behave like children, who  at one time play as wrestlers, another
time as flute players, again as  gladiators, then as trumpeters, then as tragic
actors: so  you also will be  at one time  an athlete, at another  a gladiator,
then a rhetorician,  then a philosopher, but  with your whole soul  you will be
nothing at all;  but like an ape  you imitate everything that you  see, and one
thing after  another pleases  you. For  you have  not undertaken  anything with
consideration, nor  have you  surveyed it  well; but  carelessly and  with cold
desire. Thus some  who have seen a  philosopher and having heard  one speak, as
Euphrates speaks, (and who can speak as  he does?) they wish to be philosophers
themselves also. My  man, first of all  consider what kind of thing  it is: and
then examine your own nature, if you  are able to sustain the character. Do you
wish to be a pentathlete or a wrestler? Look at your arms, your thighs, examine
your loins. For different men are formed by nature for different things. Do you
think that if you do these things, you can eat in the same manner, drink in the
same  manner, and  in the  same  manner loathe  certain things?  You must  pass
sleepless nights,  endure toil,  go away  from your kinsmen,  be despised  by a
slave, in  every thing  have the  inferior part,  in honor,  in office,  in the
courts of justice, in every little  matter. Consider these things, if you would
exchange for  them, freedom from  passions, liberty, tranquility. If  not, take
care that, like little  children, you be not now a  philosopher, then a servant
of the publicani,  then a rhetorician, then a procurator  (manager) for Caesar.
These things are not  consistent. You must be one man, either  good or bad. You
must either  cultivate your own  ruling faculty,  or external things;  you must
either exercise  your skill on internal  things or on external  things; that is
you must  either maintain  the position of  a philosopher or  that of  a common
person.
