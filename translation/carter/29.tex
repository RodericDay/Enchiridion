In every affair consider what precedes and follows, and then undertake
it. Otherwise you will begin with spirit; but not having thought of
the consequences, when some of them appear you will shamefully desist.
``I would conquer at the Olympic games.'' But consider what precedes
and follows, and then, if it is for your advantage, engage in the
affair. You must conform to rules, submit to a diet, refrain from
dainties; exercise your body, whether you choose it or not, at a stated
hour, in heat and cold; you must drink no cold water, nor sometimes
even wine. In a word, you must give yourself up to your master, as
to a physician. Then, in the combat, you may be thrown into a ditch,
dislocate your arm, turn your ankle, swallow dust, be whipped, and,
after all, lose the victory. When you have evaluated all this, if
your inclination still holds, then go to war. Otherwise, take notice,
you will behave like children who sometimes play like wrestlers, sometimes
gladiators, sometimes blow a trumpet, and sometimes act a tragedy
when they have seen and admired these shows. Thus you too will be
at one time a wrestler, at another a gladiator, now a philosopher,
then an orator; but with your whole soul, nothing at all. Like an
ape, you mimic all you see, and one thing after another is sure to
please you, but is out of favor as soon as it becomes familiar. For
you have never entered upon anything considerately, nor after having
viewed the whole matter on all sides, or made any scrutiny into it,
but rashly, and with a cold inclination. Thus some, when they have
seen a philosopher and heard a man speaking like Euphrates (though,
indeed, who can speak like him?), have a mind to be philosophers too.
Consider first, man, what the matter is, and what your own nature
is able to bear. If you would be a wrestler, consider your shoulders,
your back, your thighs; for different persons are made for different
things. Do you think that you can act as you do, and be a philosopher?
That you can eat and drink, and be angry and discontented as you are
now? You must watch, you must labor, you must get the better of certain
appetites, must quit your acquaintance, be despised by your servant,
be laughed at by those you meet; come off worse than others in everything,
in magistracies, in honors, in courts of judicature. When you have
considered all these things round, approach, if you please; if, by
parting with them, you have a mind to purchase apathy, freedom, and
tranquillity. If not, don't come here; don't, like children, be one
while a philosopher, then a publican, then an orator, and then one
of Caesar's officers. These things are not consistent. You must be
one man, either good or bad. You must cultivate either your own ruling
faculty or externals, and apply yourself either to things within or
without you; that is, be either a philosopher, or one of the vulgar.
