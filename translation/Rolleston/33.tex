Ordain for yourself forthwith a certain  principle and outline of conduct which
you may observe both when you are alone and among men.

And for the most part keep silence, or speak only what is necessary, and in few
words. But when  occasion shall require us  to speak, then we  shall speak, but
sparingly, and  not about any subject  at haphazard, nor about  gladiators, nor
horse races, nor  athletes, nor about things  to eat or drink,  which one hears
talked about everywhere, but especially not about men, as blaming, or praising,
or comparing them.

If then you are able, let your  discourse draw that of the company towards what
is fitting; but if you find yourself apart among strangers, keep silence.

Do not laugh much, nor at many things, nor unrestrainedly.

Refuse  altogether to  take an  oath,  if possible;  if  not, then  as much  as
circumstances permit.

Avoid banquets  given by  strangers and  by the  sensual But  if you  ever have
occasion to go to them, then keep your attention on the stretch that you do not
fall into  sensuality. For know that  if your companion be  corrupted, you, who
have  conversation with  him, must  needs  become corrupted  also, even  though
yourself should chance to be pure.

In things  that concern the  body, such  as food, drink,  clothing, habitation,
servants, you must  only accept what is absolutely needful.  But all that makes
for show or luxury, you must utterly proscribe.

Concerning sexual pleasures, it is right to be pure before marriage, as much as
in you  lies, but if  you do indulge  in them, let it  be according to  what is
lawful, but do not in any case make yourself disagreeable to those who use such
pleasures, nor  be fond of reproving  them, nor of putting  yourself forward as
not using them.

If you  shall be informed that  some one has been  speaking ill of you,  do not
defend yourself against his accusations,  but answer, \emph{He little knew what
other vices there are in me, or he would have said more than that.}

You need not go often to the arena; if how ever occasion should take you there,
do not  appear interested on any  man's side except  your own; that is  to say,
desire that that only may happen which  does happen, and that the conqueror may
be be who wins; for so shall you not be embarrassed. But shouting, and laughter
at this or that, and gesticulation, all this you must utterly abstain from. And
after you have gone  away, do not talk much over what has  passed, so far as it
does not tend towards your own improvement.  For from that it would appear that
you bad been impressed with the spectacle.

Do not attend everybody's recitations nor be  easily induced to go to them. But
if you do go, preserve (yet without making yourself offensive) your gravity and
tranquility.

When  you are  about to  meet any  person,  especially if  he be  one of  those
considered to be high in rank, put  before yourself what Socrates or Zeno would
have done in such a case. And then you will not fail to deal fittingly with the
occasion.

When you are going to see one of those who are great in power, imagine that you
will not find  him at home, that you  will be shut out, that the  doors will be
banged in  your face, that he  will take no notice  of you. And if  in spite of
these things it be right for you to  go, then go, and bear whatever may happen,
and  never say  to yourself  I  did not  deserve  such treatment.  For that  is
sensual, and shows that you are subject to vexation from external things.

In company, be it far from you  to bring your own doings and dangers constantly
and disproportionately into  notice. For though it is pleasant  for you to tell
of your  own dangers, yet  your adventures are  not equally pleasant  for other
persons to hear.

Be  it  far from  you  to  move laughter.  For  that  habit easily  slips  into
sensuality; and it is always enough to lower your neighbor's respect for you.

And  it  is dangerous  to  approach  to  vicious conversation.  Therefore  when
anything of the kind  may arise, rebuke him who approaches  thereto, if you can
do so opportunely. But if not, show at least by your silence, and blushing, and
serious looks that his words are disagreeable to you.
