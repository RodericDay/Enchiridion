In  everything you  do  consider what  comes  first and  what  follows, and  so
approach it. Otherwise you  will come to it with a good  heart at first because
you  have not  reflected  on  any of  the  consequences,  and afterwards,  when
difficulties have appeared, you  will desist to your shame. Do  you wish to win
at Olympia?  So do I,  by the gods,  for it is a  fine thing. But  consider the
first steps to it, and the consequences, and  so lay your hand to the work. You
must  submit  to  discipline,  eat  to order,  touch  no  sweets,  train  under
compulsion, at a fixed  hour, in heat and cold, drink no  cold water, nor wine,
except by order; you must hand yourself  over completely to your trainer as you
would to  a physician, and  then when the contest  comes you must  risk getting
hacked, and sometimes dislocate your hand,  twist your ankle, swallow plenty of
sand, sometimes get a flogging, and with  all this suffer defeat. When you have
considered all this well, then enter on the athlete's course, if you still wish
it. If you act without thought you  will be behaving like children, who one day
play at wrestlers,  another day at gladiators, now sound  the trumpet, and next
strut the stage.  Like them you will  be now an athlete, now  a gladiator, then
orator, then  philosopher, but  nothing with  all your soul.  Like an  ape, you
imitate every sight you see, and one thing after another takes your fancy. When
you  undertake  a thing  you  do  it  casually  and halfheartedly,  instead  of
considering it and looking  at it all round. In the same  way some people, when
they see a philosopher  and hear a man speaking like  Euphrates (and indeed who
can speak as he can? ), wish to be philosophers themselves.

p. 476

Man, consider  first what  it is  you are  undertaking; then  look at  your own
powers and see if you can bear it.  Do you want to compete in the pentathlon or
in wrestling? Look to your arms, your thighs, see what your loins are like. For
different men are born for different tasks.  Do you suppose that if you do this
you can live  as you do now—eat and  drink as you do now,  indulge desire and
discontent just as before?  Nay, you must sit up late,  work hard, abandon your
own people, be looked  down on by a mere slave, be ridiculed  by those who meet
you, get the worst  of it in everything—in honour, in  office, in justice, in
every  possible thing.  This is  what  you have  to consider:  whether you  are
willing to pay this price for peace  of mind, freedom, tranquillity. If not, do
not  come near;  do not  be, like  the children,  first a  philosopher, then  a
tax-collector, then an orator, then one of Caesar's procurators. These callings
do not agree.  You must be one man,  good or bad; you must  develop either your
Governing Principle,  or your  outward endowments; you  must study  either your
inner man, or outward things—in a  word, you must choose between the position
of a philosopher and that of a mere outsider.
